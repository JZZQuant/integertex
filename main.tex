\documentclass[a4paper]{article}
    \usepackage[margin=25mm]{geometry}
    \usepackage{amsmath}
    \usepackage{amsfonts}
    \usepackage{amssymb}
    \usepackage{graphicx}
    \pagenumbering{gobble}
    \usepackage{verbatim}
    \immediate\write18{texcount -tex -sum  \jobname.tex > \jobname.wordcount.tex}
    
    % Keywords command
    \providecommand{\keywords}[1]
    {
      \small	
      \textbf{\textit{Keywords---}} #1
    }
    
    \title{A Distributed Projection Algorithm for Equal Transportation Problem}
    \author{Sasikanth Goteti$^{1}$, Swapnil Kumar$^{2}$  \\
            \small $^{1}$raghavas@thoughtworks.com \\
            \small $^{2}$kswapnil@thoughtworks.com \\
    }
    \date{} % Comment this line to show today's date
    
    \begin{document}
    \maketitle
    
    \begin{abstract}
        In this article we consider a certain sub class of Integer equal flow problem ,which are known NP hard \cite{meyers}. 
        currently there exist no direct solutions for the same .Its a common problem in various inventory management systems,
        here we discuss a local minima solution which uses projection of the convex spaces to resolve the equal flows and turn 
        the problem into a known linear integer programming or constraint satisfaction problem which has reasonal know solutions that can be 
        effectively solved using simplex or other standard optimization strategies.
       \end{abstract}
    \keywords{Integer Equal Flow, Transportation,Constraint Satisfaction}
     
    \maketitle
    
    \section{Prelimenaries}
    Integer equal flow problems are known to be NP-Hard as oberserved by Meyers and Schutz\cite{meyers}.Most solutions to this would
    require graph theoretic language to formulate it correctly, as proposed by Morrison et.al \cite{Morrison2013ANS}.Simple effective
    algorithms like network simplex can be used to iteratively improve upon a simple feasible solution.We can formally define an
    integer equal flow problem as :
    \begin{equation}
        \begin{array}{rrclcl}
        \displaystyle \min_{x} & \multicolumn{3}{l}{c^T x} \\
        \textrm{s.t.} & \sum_{j:(i,j) \in A }^A x_{ij} & - & \sum_{j:(j,i) \in A } x_{ij} & = & b_i \\
        & x_{ij} & \geq & 0 & & \forall j \in N \\
        & x_{ij} & \leq & u_{ij} \\
        & x_{ik} &= & t && \forall {ik}\in Class(t)
        \end{array}
        \end{equation}
% weak theorem 1: proove that solving on numbers is same as solving on frames
% strong theorem 2:  same as above even with out of charge
% projection theorem 3; feasibility on projected system is same as feasibility on original system


    \section{Spacing}
    
    Differentials often need a bit of help with their spacing as in
    \[
        \iint xy^2\,dx\,dy 
        =\frac{1}{6}x^2y^3,
    \]
    whereas vector problems often lead to statements such as
    \[
        u=\frac{-y}{x^2+y^2}\,,\quad
        v=\frac{x}{x^2+y^2}\,,\quad\text{and}\quad
        w=0\,.
    \]

    Occasionally one gets horrible line breaks when using a list in mathematics such as 
    listing the first twelve primes  \(2,3,5,7,11,13,17,19,23,29,31,37\)\,.
    In such cases, perhaps include \verb|\mathcode`\,="213B| inside the inline maths environment so that the list breaks: \(\mathcode`\,="213B 2,3,5,7,11,13,17,19,23,29,31,37\)\,.
    Be discerning about when to do this as the spacing is different.
    
    
    
    
    
    
    \section{Arrays}
    
    Arrays of mathematics are typeset using one of the matrix environments as 
    in
    \[
        \begin{bmatrix}
            1 & x & 0 \\
            0 & 1 & -1
        \end{bmatrix}\begin{bmatrix}
            1  \\
            y  \\
            1
        \end{bmatrix}
        =\begin{bmatrix}
            1+xy  \\
            y-1
        \end{bmatrix}.
    \]
    Case statements use cases:
    \[
        |x|=\begin{cases}
            x, & \text{if }x\geq 0\,,  \\
            -x, & \text{if }x< 0\,.
        \end{cases}
    \]
    Many arrays have lots of dots all over the place as in
    \[
        \begin{matrix}
            -2 & 1 & 0 & 0 & \cdots & 0  \\
            1 & -2 & 1 & 0 & \cdots & 0  \\
            0 & 1 & -2 & 1 & \cdots & 0  \\
            0 & 0 & 1 & -2 & \ddots & \vdots \\
            \vdots & \vdots & \vdots & \ddots & \ddots & 1  \\
            0 & 0 & 0 & \cdots & 1 & -2
        \end{matrix}
    \]
    
    
    
    
    
    
    \section{Equation arrays}
    
    In the flow of a fluid film we may report
    \begin{eqnarray}
        u_\alpha & = & \epsilon^2 \kappa_{xxx} 
        \left( y-\frac{1}{2}y^2 \right),
        \label{equ}  \\
        v & = & \epsilon^3 \kappa_{xxx} y\,,
        \label{eqv}  \\
        p & = & \epsilon \kappa_{xx}\,.
        \label{eqp}
    \end{eqnarray}
    Alternatively, the curl of a vector field $(u,v,w)$ may be written 
    with only one equation number:
    \begin{eqnarray}
        \omega_1 & = &
        \frac{\partial w}{\partial y}-\frac{\partial v}{\partial z}\,,
        \nonumber  \\
        \omega_2 & = & 
        \frac{\partial u}{\partial z}-\frac{\partial w}{\partial x}\,,
        \label{eqcurl}  \\
        \omega_3 & = & 
        \frac{\partial v}{\partial x}-\frac{\partial u}{\partial y}\,.
        \nonumber
    \end{eqnarray}
    Whereas a derivation may look like
    \begin{eqnarray*}
        (p\wedge q)\vee(p\wedge\neg q) & = & p\wedge(q\vee\neg q)
        \quad\text{by distributive law}  \\
         & = & p\wedge T \quad\text{by excluded middle}  \\
         & = & p \quad\text{by identity}
    \end{eqnarray*}
    
    
    
    
    
    
    \section{Functions}
    
    Observe that trigonometric and other elementary functions are typeset 
    properly, even to the extent of providing a thin space if followed by 
    a single letter argument:
    \[
        \exp(i\theta)=\cos\theta +i\sin\theta\,,\quad
        \sinh(\log x)=\frac{1}{2}\left( x-\frac{1}{x} \right).
    \]
    With sub- and super-scripts placed properly on more complicated 
    functions,
    \[
        \lim_{q\to\infty}\|f(x)\|_q 
        =\max_{x}|f(x)|,
    \]
    and large operators, such as integrals and
    \begin{eqnarray*}
        e^x & = & \sum_{n=0}^\infty \frac{x^n}{n!}
        \quad\text{where }n!=\prod_{i=1}^n i\,,  \\
        \overline{U_\alpha} & = & \bigcap_\alpha U_\alpha\,.
    \end{eqnarray*}
    In inline mathematics the scripts are correctly placed to the side in 
    order to conserve vertical space, as in
    \(
        1/(1-x)=\sum_{n=0}^\infty x^n.
    \)
    
    
    
    
    
    
    \section{Accents}
    
    Mathematical accents are performed by a short command with one 
    argument, such as
    \[
        \tilde f(\omega)=\frac{1}{2\pi}
        \int_{-\infty}^\infty f(x)e^{-i\omega x}\,dx\,,
    \]
    or
    \[
        \dot{\vec \omega}=\vec r\times\vec I\,.
    \]
    
    
    
    
    
    \section{Command definition}
    
    \newcommand{\Ai}{\operatorname{Ai}} 
    The Airy function, $\Ai(x)$, may be incorrectly defined as this 
    integral
    \[
        \Ai(x)=\int\exp(s^3+isx)\,ds\,.
    \]
    
    \newcommand{\D}[2]{\frac{\partial #2}{\partial #1}}
    \newcommand{\DD}[2]{\frac{\partial^2 #2}{\partial #1^2}}
    \renewcommand{\vec}[1]{\boldsymbol{#1}}
    
    This vector identity serves nicely to illustrate two of the new 
    commands:
    \[
        \vec\nabla\times\vec q
        =\vec i\left(\D yw-\D zv\right)
        +\vec j\left(\D zu-\D xw\right)
        +\vec k\left(\D xv-\D yu\right).
    \]
    
    Recall that typesetting multi-line mathematics is an art normally too hard for computer recipes.  Nonetheless, if you need to be automatically flexible about multi-line mathematics, and you do not mind some rough typesetting, then perhaps invoke \verb|\parbox| to help as follows: 
    % The \verb|breqn| package is not yet reliable enough for general use.
    \newcommand{\parmath}[2][0.8\linewidth]{\parbox[t]{#1}%
        {\raggedright\linespread{1.2}\selectfont\(#2\)}}
    \[
    u_1=\parmath{ -2 \gamma  \epsilon^{2} s_{2}+\mu  \epsilon^{3} \big( \frac{3}{8} s_{2}+\frac{1}{8} s_{1} i\big)+\epsilon^{3} \big( -\frac{81}{32} s_{4} s_{2}^{2}-\frac{27}{16} s_{4} s_{2} s_{1} i+\frac{9}{32} s_{4} s_{1}^{2}+\frac{27}{32} s_{3} s_{2}^{2} i-\frac{9}{16} s_{3} s_{2} s_{1}-\frac{3}{32} s_{3} s_{1}^{2} i\big) +\int_a^b 1-2x+3x^2-4x^3\,dx }
    \]
    Also, sometimes use \verb|\parbox| to typeset multiline entries in tables.
    
    
    \section{Theorems et al.}
    
    \newtheorem{theorem}{Theorem}
    \newtheorem{corollary}[theorem]{Corollary}
    \newtheorem{lemma}[theorem]{Lemma}
    \newtheorem{definition}[theorem]{Definition}
    
    \begin{definition}[right-angled triangles] \label{def:tri}
    A \emph{right-angled triangle} is a triangle whose sides of length~\(a\), \(b\) and~\(c\), in some permutation of order, satisfies \(a^2+b^2=c^2\).
    \end{definition}
    
    \begin{lemma} 
    The triangle with sides of length~\(3\), \(4\) and~\(5\) is right-angled.
    \end{lemma}
    
    This lemma follows from the Definition~\ref{def:tri} as \(3^2+4^2=9+16=25=5^2\).
    
    \begin{theorem}[Pythagorean triplets] \label{thm:py}
    Triangles with sides of length \(a=p^2-q^2\), \(b=2pq\) and \(c=p^2+q^2\) are right-angled triangles.
    \end{theorem}
    
    Prove this Theorem~\ref{thm:py} by the algebra \(a^2+b^2 =(p^2-q^2)^2+(2pq)^2
    =p^4-2p^2q^2+q^4+4p^2q^2
    =p^4+2p^2q^2+q^4
    =(p^2+q^2)^2 =c^2\).
    
    \bibliographystyle{plain}
    \bibliography{multiincharge.bib}
    \end{document}
